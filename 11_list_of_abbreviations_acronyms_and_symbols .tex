% ==============================================
% LIST OF ABBREVIATIONS
% ==============================================
\cleardoublepage
\section*{List of Abbreviations}
\addcontentsline{toc}{section}{List of Abbreviations}
\begin{center}
\begin{tabular}{@{}p{2cm}p{\dimexpr\textwidth-4cm}@{}}
  \textbf{API} & Application Programming Interface \\
  \textbf{ARMv8} & ARM 64-bit Architecture \\
  \textbf{CLAHE} & Contrast Limited Adaptive Histogram Equalization \\
  \textbf{CNN} & Convolutional Neural Network \\
  \textbf{CRNN} & Convolutional Recurrent Neural Network \\
  \textbf{CSI} & Camera Serial Interface \\
  \textbf{CTC} & Connectionist Temporal Classification \\
  \textbf{FMCW} & Frequency-Modulated Continuous Wave \\
  \textbf{GPIO} & General-Purpose Input/Output \\
  \textbf{HC-SR04} & Ultrasonic Distance Sensor Model \\
  \textbf{ITS} & Intelligent Transportation System \\
  \textbf{IX10} & Image Sensor Output Format \\
  \textbf{LPDDR} & Low Power Double Data Rate Memory \\
  \textbf{LPR} & License Plate Recognition \\
  \textbf{MQTT} & Message Queuing Telemetry Transport \\
  \textbf{OCR} & Optical Character Recognition \\
  \textbf{pGAA} & Pixel Grid Array Format \\
  \textbf{REST} & Representational State Transfer \\
  \textbf{ROI} & Region of Interest \\
  \textbf{SDN} & Software-Defined Networking \\
  \textbf{SGBRG10} & Bayer Pattern Image Format \\
  \textbf{SMTP} & Simple Mail Transfer Protocol \\
  \textbf{SoC} & System-on-Chip \\
  \textbf{SSD} & Single Shot Detector \\
  \textbf{TPU} & Tensor Processing Unit \\
  \textbf{USB} & Universal Serial Bus \\
  \textbf{YOLO} & You Only Look Once \\
\end{tabular}
\end{center}

% ==============================================
% LIST OF SYMBOLS
% ==============================================
\cleardoublepage
\section*{List of Symbols}
\addcontentsline{toc}{section}{List of Symbols}
\begin{center}
\begin{tabular}{@{}p{2cm}p{\dimexpr\textwidth-4cm}@{}}
  $c$ & Speed of sound in air (m/s) \\
  $cm/s$ & Centimeters per second \\
  $dB$ & Decibels \\
  $fps$ & Frames per second \\
  $GHz$ & Gigahertz \\
  $kHz$ & Kilohertz \\
  $MB$ & Megabytes \\
  $MHz$ & Megahertz \\
  $m/s$ & Meters per second \\
  $px$ & Pixels \\
  $T$ & Temperature in Celsius (°C) \\
  $W$ & Watts \\
  $\mu s$ & Microseconds \\
  $^\circ C$ & Degrees Celsius \\
\end{tabular}

\printglossary

\end{center}