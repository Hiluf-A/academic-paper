Traffic management has become a critical concern in urban areas, particularly in rapidly growing cities like Addis Ababa, Ethiopia. Increasing vehicular traffic, combined with inadequate enforcement of speed limits, poses significant risks to pedestrian safety and public health. Conventional methods of traffic management rely heavily on manual oversight and static measures, often failing to address the dynamic and complex nature of urban traffic violations effectively. Recognizing these challenges, the need for advanced, technology-driven solutions has become evident.
The development of a Centralized Ultrasonic-Based Traffic Management System aims to enhance how traffic violations are monitored and penalized. By leveraging radar technology and centralized control, this system addresses limitations in current practices while offering innovative solutions, such as automated detection of speed limit violations, dynamic fine calculations, and real-time enforcement. This approach not only enhances the accuracy of traffic law enforcement but also contributes to improved road safety, reduced accident rates, and better compliance with traffic regulations.


\section{Background of the Study}
Traffic safety is a pressing issue in Addis Ababa, where pedestrian fatalities account for approximately 80\% of road traffic deaths \cite{unece2021}. Current traffic management methods rely on limited infrastructure, inadequate enforcement, and minimal use of advanced technologies, resulting in significant road safety challenges. Advanced radar-based detection systems, while effective in developed countries, are prohibitively expensive and often incompatible with the financial and infrastructural realities of cities like Addis Ababa.

In Ethiopia, there are no existing radar-based traffic enforcement systems in operation. As a result, traffic law enforcement is predominantly manual, leading to inefficiencies in detecting and addressing violations such as speeding. To fill this gap, a cost-effective, radar-based system tailored to local conditions is essential.

The proposed system seeks to address these challenges by offering an affordable yet efficient solution that combines radar technology with a centralized control system. Unlike high-cost alternatives, this system is designed to be accessible and practical for deployment in Ethiopia, focusing on real-time speed detection, automated fine calculation, and long-term scalability. By prioritizing affordability and leveraging locally available resources, the project aims to revolutionize traffic enforcement while addressing the specific needs of Addis Ababa's road safety landscape.

\section{Statement of the Problem}
Addis Ababa, the capital city of Ethiopia, faces a severe traffic safety crisis primarily driven by high pedestrian fatalities, the impact of speeding, and inadequate traffic management. With a rapidly increasing population and expanding vehicular traffic, road accidents have become a pressing issue significantly affecting public safety. Pedestrians constitute approximately 80\% of all road traffic fatalities in the city \cite{unece2021}. Speeding contributes significantly to accident severity, increasing the risk of injury by up to 3\% and serious injury or death by 5\% with each 1\% increase in average speed \cite{vital2019}. Despite various initiatives to manage traffic, like the 2017 speed management program that reduced average speeds in high-risk areas, more than 90\% of pedestrian deaths still occur at locations without dedicated walkways \cite{wikipedia2021}.

In the fiscal year ending July 2021, Ethiopia recorded 15,034 road accidents, leading to 4,161 deaths \cite{pmid2021}. Many of these accidents can be attributed to poor traffic police management, including insufficient enforcement, lack of proper training, and underreporting of incidents. Research highlights that traffic police in Addis Ababa often lack the necessary qualifications to manage accidents effectively, resulting in significant discrepancies between police-reported traffic deaths and hospital records—153 deaths reported by police compared to 84 recorded in hospitals \cite{acrs2019}. Additionally, drivers’ bad behavior, such as violent conduct and disregard for traffic regulations, remains rampant due to inadequate law enforcement.

The lack of effective traffic police management, combined with poor enforcement and insufficient data-driven approaches, exacerbates the risk of accidents in Addis Ababa. To address this, a centralized ultrasonic-based traffic management system could play a pivotal role in reducing speeding violations, automating penalty issuance, and ultimately improving road safety for both pedestrians and drivers.

\section{Objectives of the Study}

\subsection{General Objective}
To reduce traffic accidents caused by irresponsible drivers speeding in high-risk areas, ensuring safer roads through centralized control, automated enforcement, and dynamic fine systems.

\subsection{Specific Objectives}
\begin{itemize}
  \item Develop a centralized ultrasonic-based traffic management system.
  \begin{itemize}
    \item Integrate the technology speed detection, and violation tracking.
  \end{itemize}
  
  \item Implement a fine system based on speed violations.
  
  \item Enhance the system with automated penalty issuance.
  \begin{itemize}
    \item Automate the process of issuing penalties (warnings, fines, bans) for violations.
  \end{itemize}
  
  \item Implement real-time traffic violation monitoring and reporting.
  \begin{itemize}
    \item Enable continuous monitoring and reporting of traffic violations.
  \end{itemize}
\end{itemize}

\section{Scope of the Study}
The primary focus is on developing an ultrasonic-based system for detecting speeding vehicles and capturing license plate images, with enhancements aimed to streamline data management, and utomating enforcement actions.

\subsection*{1. Hardware Development}
\begin{itemize}
  \item Construction of an ultrasonic circuit capable of accurately detecting vehicle speeds.
  \item Integration of a high-resolution camera to capture images of speeding vehicles.
\end{itemize}

\subsection*{2. Software Development}
\begin{itemize}
  \item Microcontroller programming for real-time speed processing and camera triggering.
  \item Implementation of speed detection algorithms under varying traffic conditions.
  \item License Plate Recognition using OpenCV and EasyOCR with preprocessing techniques.
  \item Integration of a centralized control system using SQLite for tracking and actions.
  \item Email notification via SMTP Simple Mail Transfer Protocol to relevant authorities and systems.
\end{itemize}

\subsection*{3. System Integration}
\begin{itemize}
  \item Combining Arduino, camera, and software modules into a unified system.
  \item Ensuring reliable communication among hardware components and control system.
\end{itemize}

\subsection*{4. Centralized Control System}
\begin{itemize}
  \item The system logs violations over time.
  \item Drivers accumulating a threshold number of violations are flagged for a ban.
  \item All communication and decision-making are automated for transparency and scalability.
\end{itemize}
